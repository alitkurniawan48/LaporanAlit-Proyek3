\chapter{PENDAHULUAN}

\section{Latar Belakang}

\par
Logistik dan Manajemen Rantai Pasok merupakan sebuah elemen penting untuk praktek operasi logistik berkelanjutan \cite{grant2017sustainable}, yang bertujuan untuk menganalisa hubungan antar proses logistik baik berupa \textit{continue}, \textit{discrate}, kombinasi sampai informasi yang terkait dari titik pasokan sampai ke titik permintaan untuk dapat memenuhi permintaan pelanggan \cite{mahendrawathi2015service}. Logistik dan rantai pasokan memainkan peran penting dalam keberhasilan keseluruhan manajemen ritel \cite{fernie2018logistics}. 
\par Kualitas dalam pelayanan logstik dan manajemen rantai pasok juga dapat mempengaruhi standar kesuksesan suatu perusahaan dalam memanjakan \textit{customer} \cite{pane2019implementasi}, maka dibutuhkan inovasi terbaru yang dapat meningkatkan daya saing dalam layanan kepada konsumen, termasuk dalam mengelola data gudang. Gudang umumnya mengacu pada puluhan rak tempat penyimpanan barang, dengan peralatan penanganan material yang sesuai untuk operasi gudang kargo masuk dan keluar yang dapat menggunakan banyak lahan, memakan banyak waktu, biaya dan mampu memaksa pekerja untuk dapat bekerja dengan intensitas tinggi \cite{zhou2016warehouse}. Penggunaan data \textit{warehouse} hampir dibutuhkan oleh semua organisasi maupun perusahan karena data \textit{warehouse} dapat memungkinkan integrasi berbagai macam jenis data dari berbagai macam aplikasi atau sisitem \cite{khan2016privacy}, pada penelitian ini pengunnan teknologi integrasi data \textit{warehouse} digunakan pada alat \textit{smart conveyor} dan sistem WMS yang dapat mengatur data pemasukan, pengeluaran dan data barang permintaan terbanyak pada sebuah gudang \cite{pane2018qualitative}. Penggunaan teknologi integrasi data antara alat dan sistem dapat memaksimalkan kinerja dari \textit{warehouse managent system}. 
\par Permasalahan yang terdapat pada saat pengujian \textit{smart conveyor} dan sistem wms saat ini yaitu tidak ada sinkronisasi data yang dibaca oleh RFID dengan data yang masuk kedalam sistem wms, yang berarti data yang dibaca dengan hasil data \textit{inbound} yang ada pada sistem berbeda. Adanya kesalahan dalam mebaca data dapat menyebabkan kesalahan dalam perhitungan data pada sistem dan mengelola data dengan menggunakan algoritma K-Means dan metode \textit{moving average} \cite{awangga2018k} untuk penerapan perhitungan permintaan terbanyak.
\par Penyesesuaian intergrasi data antara data yang dibaca oleh \textit{smart conveyor} dengan data yang masuk kedalam sistem WMS dapat membenahi permasalahan dalam perhitungan data dan penerapan perhitungan permintaan terbanyak. Sehingga tidak ada lagi terjadi kesalahan dalam mengelola data gudang pada sistem \textit{warehouse management system}. \break


\section{Problems}
the problem you want to solve

\section{Objective and Contribution}
\subsection{Objective}
What your research purpose

\subsection{Contribution}
Whats your contribution

\section{Scoop and Environtment}
Scoop and environment for the research